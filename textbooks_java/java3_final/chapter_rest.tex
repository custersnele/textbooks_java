\chapter{REST}

\fcolorbox{black}[HTML]{E9F0E9}{\parbox{\textwidth}{%
\noindent \textbf{Learning goals}\\
The junior-colleague
\begin{enumerate}[nolistsep]
\item can build RESTful services in Spring Boot
\item can explain how HTTP requests are handled by Spring Boot
\item can describe what REST is
\item can explain what a REST request is
\item can explain what a REST response is
\item can provide examples of HTTP response codes
\item can design a RESTful service \cite{codecademy-rest}
\item can implement error handling for a RESTful service
\item can implement unit tests for Spring Boot controllers
\end{enumerate}}}

\section{CRUD REST API in Spring Boot}

We start this chapter with a discussion on SuperheroController.
In this class you will find the REST endpoints for CRUD operations for superheroes.

A REST controller in Spring is used to represent REST-endpoints that your application offers to outside systems.
In Spring, this can easily be achieved by annotating a class using the @RestController.
In order to map the controller to a specific http endpoint path, we will also need to configure the path using the @RequestMapping annotation.

\begin{lstlisting}[language=java, frame=single]
package be.pxl.superhero.api;

import be.pxl.superhero.service.SuperheroService;
import be.pxl.superhero.service.impl.SuperheroIdCardGenerator;
import jakarta.validation.Valid;
import org.springframework.http.HttpStatus;
import org.springframework.http.MediaType;
import org.springframework.http.ResponseEntity;
import org.springframework.web.bind.annotation.*;

import java.io.ByteArrayInputStream;
import java.util.List;

@RestController
@RequestMapping("/superheroes")
public class SuperheroController {

	private final SuperheroService superheroService;
	private final SuperheroIdCardGenerator superheroIdCardGenerator;

	public SuperheroController(SuperheroService superheroService,
							   SuperheroIdCardGenerator superheroIdCardGenerator) {
		this.superheroService = superheroService;
		this.superheroIdCardGenerator = superheroIdCardGenerator;
	}

	@GetMapping
	public List<SuperheroDTO> getSuperheroes() {
		return superheroService.findAllSuperheroes();
	}

	@GetMapping("/{superheroId}")
	public SuperheroDetailDTO getSuperheroById(@PathVariable Long superheroId) {
		SuperheroDetailDTO superheroById = superheroService.findSuperheroById(superheroId);
		return superheroById;
	}
	
	@PostMapping
	public ResponseEntity<Long> createSuperhero(@RequestBody @Valid SuperheroRequest superheroRequest) {
		return new ResponseEntity<>(superheroService.createSuperhero(superheroRequest), HttpStatus.CREATED);
	}
	
	@PutMapping("/{superheroId}")
	public SuperheroDTO updateSuperhero(@PathVariable Long superheroId, @RequestBody @Valid SuperheroRequest superheroRequest) {
		return superheroService.updateSuperhero(superheroId, superheroRequest);
	}
	
	@DeleteMapping("/{superheroId}")
	public ResponseEntity<Void> deleteSuperhero(@PathVariable Long superheroId) {
		boolean deleted = superheroService.deleteSuperhero(superheroId);
		return deleted? new ResponseEntity<>(HttpStatus.OK) : new ResponseEntity<>(HttpStatus.BAD_REQUEST);
	}

	@GetMapping(value = "/{superheroId}/idcard", produces = MediaType.APPLICATION_PDF_VALUE)
	public @ResponseBody byte[] requestSuperheroIdCard(@PathVariable Long superheroId) {

		SuperheroDetailDTO superhero = superheroService.findSuperheroById(superheroId);
		ByteArrayInputStream bis = superheroIdCardGenerator.superheroIdCard(superhero);

		return bis.readAllBytes();
	}

	@PostMapping("add-superhero-to-mission")
	public ResponseEntity<Void> addSuperheroToMission(@RequestBody SuperheroMissionRequest heroMissionRequest) {
		superheroService.addSuperheroToMission(heroMissionRequest.superheroId(), heroMissionRequest.missionId());
		return new ResponseEntity<>(HttpStatus.ACCEPTED);
	}
}
\end{lstlisting}

The class ResponseEntity is used to represent a response to an HTTP request. In other words, it is used as the return type of controller methods which serve REST requests. It is usually preferred to use a ResponseEntity object as the response because it offers us a lot of control over the response we sent to the REST endpoint caller.

Let us take a look into different HTTP methods, also called HTTP verbs, and their usage: 

\begin{itemize}
\item \textbf{GET:} The HTTP GET method is typically used to fetch items. You usually provide the identifier of the item you would like to fetch as a path variable. This is a variable that you can see in the HTTP URI, as part of the path. For example, if you would like to fetch a superhero with the identifier 212, then the URI would look like: http://localhost:8080/superheroes/212.
\item \textbf{POST:} The HTTP POST method is typically used to create new instances. The HTTP request body will contain the data for those new instances. 
\item \textbf{PUT:} The HTTP PUT method is used to update an existing instance.  The identifier of the instance to be updated is provided as a path variable, the data for the instance is contained in the HTTP request body. 
\item \textbf{DELETE:} The HTTP DELETE method is used to delete an existing instance. The object identifier should be provided as a path variable.
\end{itemize}

Other HTTP verbs are out of scope for this course.

For every HTTP verb we have an annotation to map map the HTTP POST method to a Spring Controller’s method.

The @PathVariable annotation is used to indicate that a method argument is a path variable passed by the REST endpoint caller. 

Incoming request bodies are in JSON format.  \textbf{Jackson} JSON ObjectMapper is the default HTTP Converter used by Spring.  Jackson converts the incoming JSON Request body to the Java objects.  The @RequestBody annotation is used to mark the parameter which contains the data of incoming request body.
The HTTP responses are also in JSON format. Here Jackson is reponsible for converting the Java objects to JSON responses. 
The process of converting Java objects to JSON is called \textbf{marshalling}, or serialization. When you convert JSON to Java objects it is called \textbf{unmarshalling},  or deserialization.

When Spring receives an HTTP request, the data must be validated. 
Spring validation offers  standard predefined validators.  

\begin{lstlisting}
<dependency>
	<groupId>org.springframework.boot</groupId>
	<artifactId>spring-boot-starter-validation</artifactId>
</dependency>
\end{lstlisting}

When Spring finds an argument of an HTTP method annotated with @Valid, it automatically validates the argument.  In this example, the SuperheroRequest is validated. 
The @NotBlank annotation checks that a String's trimmed length is not empty.
The @Size annotation validates that the length of the string is between min and max.

Spring also offers the annotation @Validated. This is used for group validation which is not in the scope of this course. \url{https://blog.codeleak.pl/2014/08/validation-groups-in-spring-mvc.html}

Here is a list of some common validation annotations.

\begin{tabularx}{\textwidth}{|l|X|}
\hline
@NotNull & to say that a field must not be null. \\
@NotEmpty & to say that a list field must not empty. \\
@NotBlank & to say that a string field must not be the empty string. \\
@Min and @Max & to say that a numerical field is only valid when it’s value is above or below a certain value.\\
@Size & to validate the length of a string field. \\
@Email & to say that a string field must be a valid email address.\\
\hline
\end{tabularx}

\begin{lstlisting}[language=java, frame=single]
package be.pxl.superhero.api;

import jakarta.validation.constraints.NotBlank;
import jakarta.validation.constraints.Size;

public record SuperheroRequest(String firstName,
                               @NotBlank(message = "lastname must not be blank")
                               @Size(message = "lastname must have at leat 2 characters", min = 2)
                               String lastName,
                               @NotBlank(message = "superheroname must not be blank")
                               String superheroName) {
}
\end{lstlisting}

It is also possible to use the validation constraints with the fields of java classes.

When the target argument fails the validation, Spring Boot throws a MethodArgumentNotValidException exception.

\section{Exception handling}

In Spring Boot there are several ways to convert an exception to an HTTP response status.
One of the simplest ways is to mark an exception with @ResponseStatus. 
When Spring catches the marked exception, it uses the settings provided in @ResponseStatus to create the HTTP response.

\begin{lstlisting}[language=java, frame=single]
package be.pxl.superhero.exception;

import org.springframework.http.HttpStatus;
import org.springframework.web.bind.annotation.ResponseStatus;

@ResponseStatus(HttpStatus.NOT_FOUND)
public class ResourceNotFoundException extends RuntimeException {
    public ResourceNotFoundException(String resource, String field, String value) {
        super("Not found: " + resource + " with " + field + "=" + value);
    }

    public ResourceNotFoundException(String resource, String field, long value) {
        this(resource, field, Long.toString(value));
    }
}
\end{lstlisting}

When handling the MethodArgumentNotValidException thrown by Spring validation we want to be able to manipulate the response.
In this case the solution is using a @ControllerAdvice.

\begin{lstlisting}[language=java, frame=single]
package be.pxl.superhero.api.advice;

import org.springframework.http.HttpStatus;
import org.springframework.http.ResponseEntity;
import org.springframework.web.bind.MethodArgumentNotValidException;
import org.springframework.web.bind.annotation.ControllerAdvice;
import org.springframework.web.bind.annotation.ExceptionHandler;
import org.springframework.web.bind.annotation.ResponseStatus;

import java.util.ArrayList;
import java.util.List;

@ControllerAdvice
public class ValidatedExceptionHandler {

	@ResponseStatus(HttpStatus.BAD_REQUEST)
	@ExceptionHandler(MethodArgumentNotValidException.class)
	public ResponseEntity<ErrorResponseBody> handleValidationExceptions(
			MethodArgumentNotValidException ex) {
		List<String> errors = new ArrayList<>();
		ex.getBindingResult().getAllErrors().forEach((error) -> {
			String errorMessage = error.getDefaultMessage();
			errors.add(errorMessage);
		});
		return new ResponseEntity<>(new ErrorResponseBody("Validation failed.", errors), HttpStatus.BAD_REQUEST);
	}


	private static class ErrorResponseBody {
		private final String message;
		private final List<String> errors;

		public ErrorResponseBody(String message, List<String> errors) {
			this.message = message;
			this.errors = errors;
		}

		public String getMessage() {
			return message;
		}

		public List<String> getErrors() {
			return errors;
		}
	}
}
\end{lstlisting}

We implement methods annotated with @ExceptionHandler for handling specific exception classes. 

The @ControllerAdvice is a global component in the application for error handling. The actual mechanism is simple but also very flexible: it gives us full control over the body of the response as well as the status code.  Several exceptions can be handled by the same method.

There a more possibilities for handling exceptions in Spring Boot  than the two discussed here \cite{exhand}.

\section{Unit testing}

MockMvc is defined as the main entry point for server-side Spring MVC testing.
Tests annotated with @WebMvcTest will auto-configure Spring Security and MockMvc.  These tests will also auto-configure all classes relevant to MVC tests (e.g. @Controller,  @ControllerAdvice, ...).  Using the annotation @MockBean we can provide mock objects for the @Service components.
By default, @WebMvcTest adds all @Controller beans to the application context. We can specify a subset of controllers by using the controllers attribute, as we have done in the examples discussed here. 
The method perform in the class MockMvc performs a request and returns a type that allows chaining further actions,  such as asserting expectations,  on the result.
When implementing unit tests, we pass the path parameters using MockMvcRequestBuilders and verify the status response codes and response content using MockMvcResultMatchers and MockMvcResultHandlers.

\begin{lstlisting}[language=java, frame=single]
package be.pxl.superhero.api;


import be.pxl.superhero.builder.SuperheroDetailDTOBuilder;
import be.pxl.superhero.exception.ResourceNotFoundException;
import be.pxl.superhero.service.SuperheroService;
import be.pxl.superhero.service.impl.SuperheroIdCardGenerator;
import org.junit.jupiter.api.Test;
import org.springframework.beans.factory.annotation.Autowired;
import org.springframework.boot.test.autoconfigure.web.servlet.WebMvcTest;
import org.springframework.boot.test.mock.mockito.MockBean;
import org.springframework.http.MediaType;
import org.springframework.test.web.servlet.MockMvc;
import org.springframework.test.web.servlet.request.MockMvcRequestBuilders;

import static org.mockito.Mockito.doThrow;
import static org.mockito.Mockito.when;
import static org.springframework.test.web.servlet.result.MockMvcResultHandlers.print;
import static org.springframework.test.web.servlet.result.MockMvcResultMatchers.jsonPath;
import static org.springframework.test.web.servlet.result.MockMvcResultMatchers.status;

@WebMvcTest(controllers = SuperheroController.class)
public class SuperheroControllerGetSuperheroTest {

	private static final long SUPERHERO_ID = 115L;
	public static final String FIRSTNAME = "Clark";
	public static final String LASTNAME = "Kent";
	public static final String SUPERHERO_NAME = "Superman";
	@MockBean
	private SuperheroService superheroService;
	@MockBean
	private SuperheroIdCardGenerator superheroIdCardGenerator;

	@Autowired
	private MockMvc mockMvc;

	@Test
	public void getSuperheroByIdReturnsExistingSuperhero() throws Exception {
		SuperheroDetailDTO superheroDetailDTO = SuperheroDetailDTOBuilder
				.aSuperheroDetailDTO()
				.withId(SUPERHERO_ID)
				.withFirstName(FIRSTNAME)
				.withLastName(LASTNAME)
				.withSuperheroName(SUPERHERO_NAME)
				.build();
		when(superheroService.findSuperheroById(SUPERHERO_ID))
				.thenReturn(superheroDetailDTO);
		mockMvc.perform(MockMvcRequestBuilders
						.get("/superheroes/{id}", SUPERHERO_ID)
						.accept(MediaType.APPLICATION_JSON))
				.andDo(print())
				.andExpect(status().isOk())
				.andExpect(jsonPath("$.id").value(SUPERHERO_ID))
				.andExpect(jsonPath("$.firstName").value(FIRSTNAME))
				.andExpect(jsonPath("$.lastName").value(LASTNAME))
				.andExpect(jsonPath("$.superheroName").value(SUPERHERO_NAME));
	}

	@Test
	public void getSuperheroByIdReturnsNotFoundWhenResourceNotFoundExceptionIsThrown() throws Exception {
		doThrow(ResourceNotFoundException.class)
				.when(superheroService)
				.findSuperheroById(SUPERHERO_ID);
		mockMvc.perform(MockMvcRequestBuilders
						.get("/superheroes/{id}", SUPERHERO_ID)
						.accept(MediaType.APPLICATION_JSON))
				.andExpect(status().isNotFound());
	}
}
\end{lstlisting}


\begin{lstlisting}[language=java, frame=single]
package be.pxl.superhero.api;

import be.pxl.superhero.builder.SuperheroRequestBuilder;
import be.pxl.superhero.service.SuperheroService;
import be.pxl.superhero.service.impl.SuperheroIdCardGenerator;
import com.fasterxml.jackson.databind.ObjectMapper;
import org.junit.jupiter.api.Test;
import org.mockito.ArgumentCaptor;
import org.mockito.Captor;
import org.springframework.beans.factory.annotation.Autowired;
import org.springframework.boot.test.autoconfigure.web.servlet.AutoConfigureMockMvc;
import org.springframework.boot.test.autoconfigure.web.servlet.WebMvcTest;
import org.springframework.boot.test.mock.mockito.MockBean;
import org.springframework.http.MediaType;
import org.springframework.test.web.servlet.MockMvc;
import org.springframework.test.web.servlet.request.MockMvcRequestBuilders;

import static be.pxl.superhero.builder.SuperheroRequestBuilder.FIRSTNAME;
import static be.pxl.superhero.builder.SuperheroRequestBuilder.LASTNAME;
import static be.pxl.superhero.builder.SuperheroRequestBuilder.SUPERHERO_NAME;
import static org.assertj.core.api.Assertions.assertThat;
import static org.mockito.Mockito.verify;
import static org.springframework.test.web.servlet.result.MockMvcResultMatchers.status;

@WebMvcTest(controllers = SuperheroController.class)
@AutoConfigureMockMvc
public class SuperheroControllerCreateSuperheroTest {
	@MockBean
	private SuperheroService superheroService;
	@MockBean
	private SuperheroIdCardGenerator superheroIdCardGenerator;

	@Autowired
	private MockMvc mockMvc;

	@Captor
	private ArgumentCaptor<SuperheroRequest> superheroRequestCaptor;

	@Test
	public void testLastNameIsRequired() throws Exception {
		SuperheroRequest superheroRequest = SuperheroRequestBuilder
				.aSuperheroRequest()
				.withLastName("")
				.build();
		mockMvc.perform(MockMvcRequestBuilders.post("/superheroes")
						.content(asJsonString(superheroRequest))
						.contentType(MediaType.APPLICATION_JSON)
						.accept(MediaType.APPLICATION_JSON))
				.andExpect(status().isBadRequest());
	}

	@Test
	public void testLastNameWithOneCharacterIsNotValid() throws Exception {
		SuperheroRequest superheroRequest = SuperheroRequestBuilder
				.aSuperheroRequest()
				.withLastName("A")
				.build();
		mockMvc.perform(MockMvcRequestBuilders.post("/superheroes")
						.content(asJsonString(superheroRequest))
						.contentType(MediaType.APPLICATION_JSON)
						.accept(MediaType.APPLICATION_JSON))
				.andExpect(status().isBadRequest());
	}


	@Test
	public void testSuperheroNameIsRequired() throws Exception {
		SuperheroRequest superheroRequest = SuperheroRequestBuilder
				.aSuperheroRequest()
				.withSuperheroName("")
				.build();
		mockMvc.perform(MockMvcRequestBuilders.post("/superheroes")
						.content(asJsonString(superheroRequest))
						.contentType(MediaType.APPLICATION_JSON)
						.accept(MediaType.APPLICATION_JSON))
				.andExpect(status().isBadRequest());
	}

	@Test
	public void testSuperheroIsCreatedWhenAllConstraintsSatified() throws Exception {
		SuperheroRequest superheroRequest = SuperheroRequestBuilder.aSuperheroRequest().build();
		mockMvc.perform(MockMvcRequestBuilders.post("/superheroes")
						.content(asJsonString(superheroRequest))
						.contentType(MediaType.APPLICATION_JSON)
						.accept(MediaType.APPLICATION_JSON))
				.andExpect(status().isCreated());
		verify(superheroService).createSuperhero(superheroRequestCaptor.capture());
		
		SuperheroRequest createdSuperhero = superheroRequestCaptor.getValue();
		assertThat(createdSuperhero.firstName()).isEqualTo(FIRSTNAME);
		assertThat(createdSuperhero.lastName()).isEqualTo(LASTNAME);
		assertThat(createdSuperhero.superheroName()).isEqualTo(SUPERHERO_NAME);
	}

	public static String asJsonString(final Object obj) {
		try {
			return new ObjectMapper().writeValueAsString(obj);
		} catch (Exception e) {
			throw new RuntimeException(e);
		}
	}
}
\end{lstlisting}

If you want to print the JSON result, you can use the MockMvcResultHandler.print() method.	
	
Additional examples can be found in online blogs \cite{mockmvc} or Spring documentation. 	

