\chapter{Spring Data JPA}

\fcolorbox{black}[HTML]{E9F0E9}{\parbox{\textwidth}{%
\noindent \textbf{Learning goals}\\
The junior-colleague
\begin{enumerate}[nolistsep]
\item can describe the concept of ORM.
\item can explain what JPA is, and what it’s not.
\item can denominate different JPA providers.
\item can describe what a persistence unit is.
\item can explain the fundamental interfaces of JPA.
\item can explain what the persistence context is.
\item can implement entity classes.
\item can describe the entity objects’ lifecycle.
\item can implement different types of relationships between entity classes.
\item can implement CRUD operations.
\item can implement queries.
\item can implement named queries.
\item can explain, identify and solve the N + 1 query problem.
\end{enumerate}}}

  
\section{What is ORM?}

Object-Relational Mapping (ORM) is a technique that lets you query and manipulate data from a relational database using an object-oriented programming language.

\includegraphics[width=\textwidth]{./images/chapter6/orm}


\section{What is JPA?}

The Java Persistence API (JPA; recently renamed to Jakarta Persistence API) is a specification that defines how to persist data in Java applications. JPA mainly focuses on the ORM layer and managing persistent objects.

JPA is a specification which means JPA consists of implementation guidelines for the Java ORM layer and the syntax. The specification only comes with interfaces, no actual implementation.  A reference implementation can be provided but other companies can create and distribute their own implementation of the specification.

\frame{\includegraphics[width=\textwidth]{./images/chapter6/entity_class_specification}}

For our exercises and projects we will use Hibernate as the actual implementation of de JPA specification \footnote{\url{https://hibernate.org/ and https://hibernate.org/orm/}}.  

Spring Data JPA adds a layer on top of JPA. It uses all the features defined by the JPA specification, but adds no-code implementation of the repository pattern and the creation of database queries from method names.

The JpaRepository interface is an extension of CrudRepository.
JpaRepository extends PagingAndSortingRepository which in turn extends CrudRepository.

Their main functions are:

\begin{itemize}
\item CrudRepository mainly provides CRUD functions.
\item PagingAndSortingRepository provides methods to do pagination and sorting records.
\item JpaRepository provides some JPA-related methods such as flushing the persistence context and deleting records in a batch.
\end{itemize}


\section{Datasource}

In Spring Boot a DataSource-object is the preferred means of getting a connection to a database.
The interface jakarta.sql.DataSource is implemented by the database driver vendor. 

\includegraphics[width=\textwidth]{./images/chapter-jpa/springdatajpa}

The datasource can be specified in the application.properties file.
Here are some common database properties:

\begin{tabular}{|l|p{8cm}|}
\hline
spring.datasource.url & JDBC URL of the database.\\
spring.datasource.username & Login username of the database.\\
spring.datasource.password & Login password of the database.\\
spring.jpa.show-sql & Whether to enable logging of SQL statements. Default: false\\
spring.jpa.hibernate.ddl-auto & Possible values: none (production), create, create-drop, validate and update. \footnote{\url{https://stackoverflow.com/questions/42135114/how-does-spring-jpa-hibernate-ddl-auto-property-exactly-work-in-spring}}\\
\hline
\end{tabular}

Alternatively, the data source configuration can be programmatically.

The appropriate JDBC driver for your database must be included in your project. This is achieved by declaring the driver as a dependency in your project's Maven configuration file, pom.xml. The dependency ensures that the JDBC driver is available during runtime, allowing Spring Data JPA to establish and manage database connections.

Here is an example of how to include a JDBC driver for MySQL in your pom.xml file:

\begin{lstlisting}
<dependency>
<groupId>com.mysql</groupId>
<artifactId>mysql-connector-j</artifactId>
<scope>runtime</scope>
</dependency>
\end{lstlisting}

This configuration snippet includes the MySQL JDBC driver, mysql-connector-j.

Similarly,  to connect to a PostgreSQL database, you would use the PostgreSQL JDBC driver as shown below:

\begin{lstlisting}
<dependency>
<groupId>org.postgresql</groupId>
<artifactId>postgresql</artifactId>
<scope>runtime</scope>
</dependency>
\end{lstlisting}

Adjusting the driver version in your pom.xml may be necessary as you upgrade your database server or migrate to a newer version of Spring Data JPA.


\section{The Entity class}

\begin{lstlisting}[frame=single,language=java]
import jakarta.persistence.Entity;
import jakarta.persistence.Table;
import jakarta.persistence.Column;
import jakarta.persistence.Id;
import jakarta.persistence.GeneratedValue;
import jakarta.persistence.GenerationType;
import jakarta.persistence.Embedded;
import jakarta.persistence.Embeddable;

@Entity
@Table(name = "events")
public class Event {

    @Id
    @GeneratedValue(strategy = GenerationType.IDENTITY)
    private Long id;

    @Column(name = "name", nullable = false,  length = 200)
    private String title;

    @Embedded
    private EventDetails details;

    // Constructors, getters, and setters
}

@Embeddable
class EventDetails {
    private LocalDate date;
    @Column(length = 512)
    private String location;

    // Constructors, getters, and setters
}
\end{lstlisting}

A JPA entity class is a POJO (Plain Old Java Object) class  that is annotated as having the ability to represent objects in the database.
The \textbf{@Entity} annotation is used to declare that a class is an entity, so JPA will manage it and map it to a database table.

The \textbf{@Id} annotation marks a field as a primary key field.

The \textbf{@GeneratedValue} annotation is used to configure the primary key generation strategy for an entity. This annotation is usually combined with @Id to specify that the persistence provider should automatically generate a unique identifier for the entity objects. There are several strategies defined in the GenerationType enumeration that can be used with @GeneratedValue. Here's an overview:

\begin{itemize}
\item \textbf{GenerationType.AUTO}

This is the default strategy and the persistence provider will choose the generation strategy based on the specific database capabilities and dialect. 

\item \textbf{GenerationType.IDENTITY}

Indicates that the persistence provider must assign primary keys using the database identity column. This is typically supported by SQL databases with an auto-increment column.

\item \textbf{GenerationType.SEQUENCE}

Specifies that the primary key values will be generated using a database sequence. This is a special database object that generates numbers in sequential order. Not all databases support sequences.
This strategy is useful for databases supporting sequences, like Oracle, PostgreSQL, etc. It's beneficial for high-volume applications due to better performance compared to IDENTITY, as the sequence generation can be more efficiently managed.

\item \textbf{GenerationType.TABLE}

Uses a database table to simulate a sequence. This strategy uses a table to hold the next id incrementally. It's a portable solution but not as efficient as using database-specific features like sequences or identity columns.
\end{itemize}

The \textbf{@Column} annotation is used to specify the details of the column to which a field or property will be mapped. You can use column annotation with the following most commonly used attributes:
\begin{itemize}
\item \textbf{name}: specify the name of the column.
\item \textbf{length}: specify the size of thee column, particularly for a String value.
\item \textbf{nullable}: mark the column to be NOT NULL when the database schema is generated.
\item \textbf{unique}: mark the column to contain only unique values.
\end{itemize}


The \textbf{@Embeddable} annotation marks a class to be embedded within another entity.
In this example,  the EventDetails does not need its own table; instead, its properties are incorporated into the table of the entity that embeds it.

The annotation \textbf{@Embedded}  is used to denote a class whose instances are stored as an intrinsic part of an owning entity.
 All the fields of EventDetails class are embedded within the events table, avoiding the need for a separate table.


\section{Entity lifecycle}

The EntityManager is a core interface of JPA.  An instance of EntityManager is used to create and remove  entity instances, to find entities by their primary key,  and to query over entities.  In fact,  an instance of the EntityManager is used to interact with the persistence context. 

The persistence context is one of the main concepts in JPA.
It is a set of all entity object that you are currently using or used recently. You can think of the persistence context as some kind of first-level cache. Each entity object in the persistence context represents a record in the database.
The persistence context manages these entity objects and their lifecycle. Each entity object has one of 4 states: new, managed, removed, and detached.

\includegraphics[width=\textwidth]{./images/chapter6/entity_states}

A newly instantiated entity object is in state \textbf{new} or \textbf{transient}. The entity object hasn't been persisted yet, so there's no database record yet. The persistence context does not know about the entity object yet. 

All entity objects that attached to the persistence context are in the lifecycle state \textbf{managed}. An entity object becomes managed when it is persisted to the database. Entity object retrieved from the database are also in the managed state.
If a managed entity object is modified (updated) within an active transaction, the change is detected by the persistence context and passed on to the database.

When a managed entity is removed within an active transaction, the state changes from managed to removed, and the record is physically deleted from the database (when the transaction commits).

An entity gets detached when it is no longer managed by the persistence context but still represents a record in the database.
Detached objects are limited in functionality.

 
Programmatically we need to use the entitymanager to change the state of the entity object in the persistence context which results in changes or updates in our database. 

When using Spring Data JPA you don't have to implement the interaction with the entitymanager. When you create a Repository, the SimpleJpaRepository class provided by Spring Data JPA provides the default implementation. SimpleJpaRepository class internally uses JPA entitymanager.

\begin{oefening}
Open the class SimpleJpaRepository and take a look at its implementation (e.g. save()-method).
\end{oefening}


\section{Queries}

With Spring Data JPA you can create query methods to select records from the database without writing SQL queries. Behind the scenes, Spring Data JPA will create SQL queries based on the query method and execute the query for us.

The query generation from the method name is a query generation strategy where the invoked query is derived from the name of the query method.

Consider a UserRepository with query method findByEmailAddressAndLastname().

\begin{lstlisting}
public interface UserRepository extends JpaRepository<User, Long> {
  List<User> findByEmailAddressAndLastname(String emailAddress, String lastname);
}
\end{lstlisting}

Spring Data JPA behind the scene creates a query which translates into the JPQL query: select u from User u where u.emailAddress = ?1 and u.lastname = ?2. 

\begin{oefening}
In the interface SuperheroRepository you add a method for retrieving a Superhero by its Supheroname. The method should an Optional.
\end{oefening}


\section{Unit testing a repository}

Spring Data JPA offers an annotation @DataJpaTest which makes repository testing possible with a minimum of configuration. 

Add the h2 dependency with scope test in your pom.xml. This way the unit test for your repository will use the h2 database to test your queries.

\begin{lstlisting}
<dependency>
	<groupId>com.h2database</groupId>
	<artifactId>h2</artifactId>
	<scope>test</scope>
</dependency>
\end{lstlisting}

The builder pattern is used in this example to create entity objects for testing purpose. These entity objects are stored in the in-memory database. There is an IntelliJ IDEA plugin that generates the builder-classes for you (Generate Builder plugin).

The test uses the entity managers, which is injected in the test with the @PersistenceContext annotation. The flush() forces to synchronize the persistence context to the database. The clear() empties the persistence context. Therefore, all entities are detached and can be fetched again from the database.

\begin{lstlisting}[frame=single, language=java]
package be.pxl.superhero.repository;

import be.pxl.superhero.builder.SuperheroBuilder;
import be.pxl.superhero.domain.Superhero;
import jakarta.persistence.EntityManager;
import jakarta.persistence.PersistenceContext;
import org.junit.jupiter.api.BeforeEach;
import org.junit.jupiter.api.Test;
import org.springframework.beans.factory.annotation.Autowired;
import org.springframework.boot.test.autoconfigure.orm.jpa.DataJpaTest;

import java.util.Optional;

import static org.junit.jupiter.api.Assertions.assertEquals;
import static org.junit.jupiter.api.Assertions.assertTrue;

@DataJpaTest
public class SuperheroRepositoryTest {

	private static final String SUPERHERO_NAME = "Superman";

	@PersistenceContext
	protected EntityManager entityManager;

	@Autowired
	private SuperheroRepository superheroRepository;

	private final Superhero superhero = SuperheroBuilder.aSuperhero()
			.withFirstName("Clark")
			.withLastName("Kent")
			.withSuperheroName(SUPERHERO_NAME)
			.build();

	@BeforeEach
	public void init() {
		superheroRepository.save(superhero);
		entityManager.flush();
		entityManager.clear();
	}

	@Test
	public void returnsSuperheroWithGivenSuperheroName() {
		Optional<Superhero> superheroUnderTest = superheroRepository.findSuperheroBySuperheroName(SUPERHERO_NAME);

		assertTrue(superheroUnderTest.isPresent());
		assertEquals(SUPERHERO_NAME, superheroUnderTest.get().getSuperheroName());

	}

	@Test
	public void returnsEmptyOptionalWhenNoInstituteWithGivenName() {
		Optional<Superhero> superheroUnderTest = superheroRepository.findSuperheroBySuperheroName("Robin Hood");

		assertTrue(superheroUnderTest.isEmpty());
	}
}
\end{lstlisting}

\begin{oefening}
You should be able to find a mission by it's name from the database. Add this query in the MissionRepository and test the query.
\end{oefening}


\section{Relationships}

From a database perspective, table relationships can be:
\begin{itemize}
\item one-to-many
\item one-to-one
\item many-to-many
\end{itemize}

JPA defines multiple annotations for mapping table relationships. For the superhero-backend we'll use a many-to-many relationship between superheroes and missions. A superhero can participate in multiple missions and multiple superheroes bundle forces in one mission. A link table is necessary to store the data.

Relationships can be either unidirectional of bidirectional. Only one party in a relationship is the owner of the relationship. 
 

\subsection{Many-to-many bidirectional relationship}

The Superhero class is owner of the relationship between Mission and Superhero. Because the relationship is bidirectional, we always maintain a set of a superhero's missions and in a mission we maintain a set of the superheroes participating.
When you add a mission to a superhero, you have to make sure the superhero is also added to the mission's list of superheroes.
Study the code below.


\begin{lstlisting}
package be.pxl.superhero.domain;

import jakarta.persistence.*;
import org.apache.logging.log4j.LogManager;
import org.apache.logging.log4j.Logger;

import java.util.ArrayList;
import java.util.List;

@Entity
@Table(name="superheroes")
public class Superhero {

	private static final Logger LOGGER = LogManager.getLogger(Superhero.class);

	@Id
	@GeneratedValue(strategy = GenerationType.IDENTITY)
	private Long id;
	
	private String firstName;
	private String lastName;
	@Column(unique = true)
	private String superheroName;
	@Column(name="notes")
	private String description;
	@ManyToMany
	private List<Mission> missions = new ArrayList<>();

	public Superhero() {
		// JPA only
	}

	public Superhero(String firstName, String lastName, String superheroName) {
		if (LOGGER.isDebugEnabled()) {
			LOGGER.debug("Creating a new superhero...");
		}
		this.firstName = firstName;
		this.lastName = lastName;
		this.superheroName = superheroName;
	}

	public Long getId() {
		return id;
	}

	public String getFirstName() {
		if (LOGGER.isTraceEnabled()) {
			LOGGER.trace(String.format("FirstName of %s was revealed.", superheroName));
		}
		return firstName;
	}

	public void setFirstName(String firstName) {
		this.firstName = firstName;
	}

	public String getLastName() {
		if (LOGGER.isFatalEnabled()) {
			LOGGER.fatal(String.format("LastName of %s was revealed.", superheroName));
		}
		return lastName;
	}

	public void addMission(Mission mission) {
		if (mission.isCompleted()) {
			throw new IllegalArgumentException("This mission was already completed.");
		}
		if (onMission(mission)) {
			throw new IllegalArgumentException("This mission was already assigned.");
		}
		missions.add(mission);
		mission.addSuperhero(this);
	}

	public boolean onMission(Mission mission) {
		return missions.stream().anyMatch(m -> m.getName().equals(mission.getName()));
	}

	public List<Mission> getMissions() {
		return missions;
	}

	public void setLastName(String lastName) {
		this.lastName = lastName;
	}

	public String getSuperheroName() {
		return superheroName;
	}

	public void setSuperheroName(String superheroName) {
		this.superheroName = superheroName;
	}

	@Override
	public String toString() {
		return superheroName;
	}
}
\end{lstlisting}

\begin{lstlisting}
package be.pxl.superhero.domain;

import jakarta.persistence.*;

import java.util.ArrayList;
import java.util.List;

@Entity
public class Mission {
	@Id
	@GeneratedValue(strategy = GenerationType.IDENTITY)
	private Long id;

	@Column(unique = true)
	private String name;

	private boolean completed;

	@ManyToMany(mappedBy = "missions")
	private List<Superhero> superheroes = new ArrayList<>();

	public Long getId() {
		return id;
	}

	public String getName() {
		return name;
	}

	public void setName(String name) {
		this.name = name;
	}

	public boolean isCompleted() {
		return completed;
	}

	public void setCompleted(boolean completed) {
		this.completed = completed;
	}

	public void addSuperhero(Superhero superhero) {
		superheroes.add(superhero);
	}

	public List<Superhero> getSuperheroes() {
		return superheroes;
	}


}
\end{lstlisting}


\begin{oefening}
Create the bidirectional many-to-many relationship between superhero and mission in your project. 
Write unit tests for the methode addMission() in the class Superhero.
\end{oefening}

\section{Transactions}

Next, we create the REST-endpoint to add a mission to a superhero.

\includegraphics[width=\textwidth]{./images/chapter-jpa/superhero_controller_add_superhero_to_mission}

\begin{lstlisting}
package be.pxl.superhero.service.impl;

import be.pxl.superhero.api.SuperheroDTO;
import be.pxl.superhero.api.SuperheroRequest;
import be.pxl.superhero.domain.Mission;
import be.pxl.superhero.domain.Superhero;
import be.pxl.superhero.exception.ResourceNotFoundException;
import be.pxl.superhero.repository.MissionRepository;
import be.pxl.superhero.repository.SuperheroRepository;
import be.pxl.superhero.service.SuperheroService;
import org.springframework.stereotype.Service;
import org.springframework.transaction.annotation.Transactional;

import java.util.List;

@Service
public class SuperheroServiceImpl implements SuperheroService {

    private final SuperheroRepository superheroRepository;
    private final MissionRepository missionRepository;

    public SuperheroServiceImpl(SuperheroRepository superheroRepository,
                                MissionRepository missionRepository) {
        this.superheroRepository = superheroRepository;
        this.missionRepository = missionRepository;
    }

   
    // other methods not included 

    @Transactional
    public void addSuperheroToMission(Long superheroId, Long missionId) {
        Superhero superhero = superheroRepository.findById(superheroId).orElseThrow(() -> new ResourceNotFoundException("Superhero", "ID", superheroId));
        Mission mission = missionRepository.findById(missionId).orElseThrow(() -> new ResourceNotFoundException("Mission", "ID", missionId));
        superhero.addMission(mission);
    }
}
\end{lstlisting}

In the service-layer we retrieve our superhero and mission entity objects (and throw an exception if they don't exist). Then, the mission entity object is added to the superhero's mission and otherwise. Thanks to the @Transactional annotation, all this runs in one transaction. When the transaction is commited, the updated superhero entity is synchronized with the database and the changes are stored in the database. In other words, a record is saved in the link table when no exception occurs. If an exception occurs, the transaction is rolled back.


Lastly, we need DTO's with extra fields with the detailed superhero and mission information.

\includegraphics{./images/chapter-jpa/superhero_detail_dto}


Overview of spring data jpa annotations.

Entity Definition
@Entity: Specifies that a class is an entity and is mapped to a database table.
@Table: Specifies the table in the database with which the entity is mapped.
@Column: Specifies the mapped column for a persistent property or field.
@Id: Marks a field as a primary key.
@GeneratedValue: Specifies the strategy for generating primary key values.
@Embedded: Indicates a field should be treated as an embeddable type.
@Embeddable: Marks a class to be embedded within an entity.
Relationships
@ManyToOne: Defines a many-to-one relationship between two entities.
@OneToMany: Defines a one-to-many relationship between two entities.
@OneToOne: Defines a one-to-one relationship between two entities.
@ManyToMany: Defines a many-to-many relationship between two entities.
@JoinColumn: Specifies a column for joining an entity association or element collection.
Inheritance
@Inheritance: Defines the inheritance strategy to be used for an entity class hierarchy.
Fetching Strategy
@Fetch: Specifies a custom fetching strategy for an association.
@LazyCollection: Specifies that a collection should be loaded lazily.
Miscellaneous
@Transient: Marks a field as not persistent.
@Temporal: Converts between Java's Date and Time types and SQL's DATE, TIME, and TIMESTAMP types.
@Enumerated: Specifies that a persistent property or field should be persisted as an enumerated type.
Repository Support
@Repository: Indicates that an annotated class is a repository, which is an abstraction of data access and storage.
@Query: Specifies a SQL or JPQL query directly on the repository method.
@Modifying: Indicates that a query method should be considered as modifying query (such as UPDATE, DELETE).
Transaction Management
@Transactional: Specifies that a method or class must be executed within a transaction.



