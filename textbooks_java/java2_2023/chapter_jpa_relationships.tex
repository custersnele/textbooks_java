\chapter{Database relaties met Spring Data JPA}


\fcolorbox{black}[HTML]{E9F0E9}{\parbox{\textwidth}{%
\noindent \textbf{Doelstellingen}\\
De junior-collega
\begin{enumerate}[nolistsep]
\item Kan entiteitsklassen implementeren.
\item Kan verschillende soorten relaties tussen entiteitsklassen implementeren: on-ton, many-to-one and many-to-many op een unidirectionele en bidirectionele manier.
\item Kan eenvoudige queries implementeren.
\item Kan het N + 1 queryprobleem uitleggen, identificeren en oplossen.
\item Kan 'lazy loading' gebruiken, zelfs als OSIV niet is ingeschakeld.
\end{enumerate}}}

\section{Relationships}

Spring Data JPA ondersteunt relationele database associaties:

\begin{itemize}
\item one-to-one associaties,
\item many-to-one associaties en
\item many-to-many associaties.
\end{itemize}
Je kan de associaties zowel uni- als bidirectional maken.


\subsection{One-to-One associatie}

In een één-op-één-relatie is één record in een tabel geassocieerd met slechts één record in een andere tabel.

\begin{lstlisting}[frame=single, language=java]
package be.pxl.demo.domain;

import jakarta.persistence.*;

@Entity
@Table(name = "contact_info")
public class ContactInformation {
    @Id
    @GeneratedValue(strategy = GenerationType.IDENTITY)
    private Long id;
    private String phone;
    private String email;
    private String linkedIn;

    public Long getId() {
        return id;
    }

    public String getPhone() {
        return phone;
    }

    public void setPhone(String phone) {
        this.phone = phone;
    }

    public String getEmail() {
        return email;
    }

    public void setEmail(String email) {
        this.email = email;
    }

    public String getLinkedIn() {
        return linkedIn;
    }

    public void setLinkedIn(String linkedIn) {
        this.linkedIn = linkedIn;
    }

    @Override
    public String toString() {
        return "ContactInformation{" +
                "id=" + id +
                ", phone='" + phone + '\'' +
                ", email='" + email + '\'' +
                ", linkedIn='" + linkedIn + '\'' +
                '}';
    }
}
\end{lstlisting}

Een one-to-one relatie resulteert in een foreign key in de tabel die eigenaar (owner) is van de associatie.
De instantievariabelen met het datatype van de geassocieerde entiteitsklasse wordt geannoteerd met @OneToOne.  Je kunt de naam van de kolom met de foreign key aanpassen met de annotatie @JoinColumn.

\begin{lstlisting}[frame=single, language=java]
package be.pxl.demo.domain;

import jakarta.persistence.*;

@Entity
public class Researcher {

	@Id
	@GeneratedValue(strategy = GenerationType.IDENTITY)
	private Long id;
	@Column(length = 40, nullable = false)
	private String firstname;
	@Column(length = 40, nullable = false)
	private String lastname;
	@OneToOne(cascade = CascadeType.ALL)
	private ContactInformation contactInformation;

	public Long getId() {
		return id;
	}

	public String getFirstname() {
		return firstname;
	}

	public void setFirstname(String firstname) {
		this.firstname = firstname;
	}

	public String getLastname() {
		return lastname;
	}

	public void setLastname(String lastname) {
		this.lastname = lastname;
	}

	public ContactInformation getContactInformation() {
		return contactInformation;
	}

	public void setContactInformation(ContactInformation contactInformation) {
		this.contactInformation = contactInformation;
	}

	@Override
	public String toString() {
		return "Researcher{" +
				"id=" + id +
				", firstname='" + firstname + '\'' +
				", lastname='" + lastname + '\'' +
				", contactInformation=" + contactInformation +
				'}';
	}
}

\end{lstlisting}

De betekenis van CascadeType.ALL is dat de persistentiecontext alle EntityManager-operaties (PERSIST, REMOVE, REFRESH, MERGE, DETACH) zal doorgeven (cascaderen) naar de gerelateerde entiteiten. Dit betekent dat wanneer een operatie wordt uitgevoerd op de hoofdentiteit, deze operatie automatisch ook wordt toegepast op alle gerelateerde entiteiten. Bijvoorbeeld, als een hoofdentiteit wordt opgeslagen, worden alle gerelateerde entiteiten ook opgeslagen. Hetzelfde geldt voor verwijderen, vernieuwen, samenvoegen en loskoppelen. Dit zorgt voor een consistent beheer van de relaties tussen entiteiten in de database en vereenvoudigt de transactiebeheer aanzienlijk.

\subsection{Many-to-one relatie}

Stel dat er meerdere onderzoekers werken aan één project, maar elke onderzoeker is toegewezen aan één project.

Hier is de entiteitsklasse Project.

\begin{lstlisting}[frame=single, language=java]
package be.pxl.demo.domain;

import jakarta.persistence.*;

import java.time.LocalDate;
import java.util.ArrayList;
import java.util.List;

@Entity
public class Project {
	@Id
	@GeneratedValue(strategy = GenerationType.IDENTITY)
	private Long id;
	private String name;
	private LocalDate start;
	@Enumerated(value= EnumType.STRING)
	private ProjectPhase projectPhase;

	public Project() {
	    // JPA only
	}

	public Project(String name) {
		this.name = name;
		this.start = LocalDate.now();
		this.projectPhase = ProjectPhase.INITIATING;
	}

	public Long getId() {
		return id;
	}

	public String getName() {
		return name;
	}

	public void setName(String name) {
		this.name = name;
	}

	public LocalDate getStart() {
		return start;
	}

	public void setStart(LocalDate start) {
		this.start = start;
	}

	public ProjectPhase getProjectPhase() {
		return projectPhase;
	}

	public void setProjectPhase(ProjectPhase projectPhase) {
		this.projectPhase = projectPhase;
	}

	@Override
	public boolean equals(Object o) {
		if (this == o) {
			return true;
		}
		if (o == null || getClass() != o.getClass()) {
			return false;
		}

		Project project = (Project) o;

		return id != null ? id.equals(project.id) : project.id == null;
	}

	@Override
	public int hashCode() {
		return id != null ? id.hashCode() : 0;
	}

	@Override
	public String toString() {
		return name;
	}
}

\end{lstlisting}

\begin{lstlisting}[frame=single, language=java]
package be.pxl.paj.domain;

public enum ProjectPhase {
	INITIATING,
	PLANNING,
	EXECUTING,
	CLOSING;
}
\end{lstlisting}

De meest voorkomende optie om een enum-waarde in JPA te mappen naar en vanuit zijn database-representatie is het gebruik van de @Enumerated annotatie. Op deze manier zal de JPA-provider de enum-waarde  omzetten naar de ordinale waarde of String-waarde. Wanneer we de String-waarde gebruiken, kunnen we veilig nieuwe enum-waarden toevoegen of de volgorde van onze enum-waarden wijzigen. Het hernoemen van een enum-waarde zal echter nog steeds de databasedata breken.

Om de relatie te creëren tussen een onderzoeker en het project waaraan hij werkt, voegen we een @ManyToOne relatie toe in de entiteitsklasse Researcher.

\begin{lstlisting}[frame=single,language=java]
package be.pxl.demo.domain;

import jakarta.persistence.*;
import org.apache.logging.log4j.LogManager;
import org.apache.logging.log4j.Logger;

@Entity
public class Researcher {

	private static final Logger LOGGER = LogManager.getLogger(Researcher.class);

	@Id
	@GeneratedValue(strategy = GenerationType.IDENTITY)
	private Long id;
	@Column(length = 40, nullable = false)
	private String firstname;
	@Column(length = 40, nullable = false)
	private String lastname;
	@OneToOne(cascade = CascadeType.ALL)
	private ContactInformation contactInformation;
	@ManyToOne
	private Project project;

	public Long getId() {
		return id;
	}

	public String getFirstname() {
		return firstname;
	}

	public void setFirstname(String firstname) {
		this.firstname = firstname;
	}

	public String getLastname() {
		return lastname;
	}

	public void setLastname(String lastname) {
		this.lastname = lastname;
	}

	public ContactInformation getContactInformation() {
		return contactInformation;
	}

	public void setContactInformation(ContactInformation contactInformation) {
		this.contactInformation = contactInformation;
	}

	public void setProject(Project project) {
		if (this.project != null) {
			if (this.project.equals(project)) {
				LOGGER.info("Researcher [" + id + "] already assigned to [" + project.getName() + "]");
				return;
			}
		}
		this.project = project;
	}

	@Override
	public String toString() {
		return String.format("%s %s (%d)", firstname, lastname, id);
	}
}
\end{lstlisting}

Op dit moment is het niet mogelijk om te zien welke onderzoekers aan een project werken. Om dit mogelijk te maken moeten we de relatie tussen project en onderzoeker bidirectioneel maken. Onderzoeker wordt gedefinieerd als de \textbf{eigenaar} of owner van de relatie. Om deze associatie bidirectioneel te maken, moeten we de entiteit waarnaar verwezen wordt definiëren.  Bij de inverse relatie mappen we naar de eigenaarskant van de relatie.  De waarde van mappedBy is de naam van het attribuut voor associatie-mapping aan de eigenaarskant.  De manier om deze gerelateerde data op te halen is standaard \textbf{lazy}.  Dit betekent dat de onderzoekers voor een project alleen uit de database worden opgehaald wanneer de methode getResearchers() wordt aangeroepen.


\begin{lstlisting}[frame=single, language=java]
package be.pxl.demo.domain;

import jakarta.persistence.*;

import java.time.LocalDate;
import java.util.ArrayList;
import java.util.List;

@Entity
public class Project {
	@Id
	@GeneratedValue(strategy = GenerationType.IDENTITY)
	private Long id;
	private String name;
	private LocalDate start;
	@OneToMany(mappedBy = "project")
	private List<Researcher> researchers = new ArrayList<>();
	@Enumerated(value= EnumType.STRING)
	private ProjectPhase projectPhase;

	public Project() {
	}

	public Project(String name) {
		this.name = name;
		this.start = LocalDate.now();
		this.projectPhase = ProjectPhase.INITIATING;
	}

	public Long getId() {
		return id;
	}

	public String getName() {
		return name;
	}

	public void setName(String name) {
		this.name = name;
	}

	public LocalDate getStart() {
		return start;
	}

	public void setStart(LocalDate start) {
		this.start = start;
	}

	public List<Researcher> getResearchers() {
		return researchers;
	}

	public ProjectPhase getProjectPhase() {
		return projectPhase;
	}

	public void setProjectPhase(ProjectPhase projectPhase) {
		this.projectPhase = projectPhase;
	}

	@Override
	public boolean equals(Object o) {
		if (this == o) {
			return true;
		}
		if (o == null || getClass() != o.getClass()) {
			return false;
		}

		Project project = (Project) o;

		return id != null ? id.equals(project.id) : project.id == null;
	}

	@Override
	public int hashCode() {
		return id != null ? id.hashCode() : 0;
	}

	public void addResearcher(Researcher researcher) {
		researchers.add(researcher);
	}

	public void removeResearcher(Researcher researcher) {
		researchers.remove(researcher);
	}

	@Override
	public String toString() {
		return name;
	}
}
\end{lstlisting}

Bij het gebruik van een bidirectionele relatie moeten beide associaties binnen de relatie zorgvuldig worden beheerd.
Daarom moeten we de methode \textit{setProject()} in de klasse Researcher herschrijven.

\begin{lstlisting}
public void setProject(Project project) {
	if (this.project != null) {
		if (this.project.equals(project)) {
			LOGGER.info("Researcher [" + id + "] already assigned to [" + project.getName() + "]");
			return;
		}
		this.project.removeResearcher(this);
	}
	this.project = project;
	project.addResearcher(this);
}
\end{lstlisting}
 
Het is geen goed idee om de klasse Project eigenaar te maken van de relatie tussen projecten en onderzoekers.  Je zult dan een inefficiënt databaseschema krijgen.  Meer informatie is te vinden in deze post:

\url{https://vladmihalcea.com/the-best-way-to-map-a-onetomany-association-with-jpa-and-hibernate/}.


\subsection{Many-to-many relaties}

Eigenlijk willen we dat onze onderzoekers aan meerdere projecten werken. Daarom moeten we een @ManyToMany relatie introduceren tussen de klasse Researcher en Project."

\begin{lstlisting}[frame=single,language=java]
package be.pxl.demo.domain;

import org.apache.logging.log4j.LogManager;
import org.apache.logging.log4j.Logger;

import jakarta.persistence.*;
import java.util.ArrayList;
import java.util.List;

@Entity
public class Researcher {

	// all other properties left out intentionally
	
	@ManyToMany
	private List<Project> projects = new ArrayList<>();


	public void addProject(Project project) {
		if (this.projects.contains(project)) {
				LOGGER.info("Researcher [" + id + "] already assigned to [" + project.getName() + "]");
				return;
		}
		this.projects.add(project);
		project.addResearcher(this);
	}

	public List<Project> getProjects() {
		return projects;
	}

	// all other methods left out intentionally
}
\end{lstlisting}

Er zal nu een koppeltabel ontstaant waar we wordt bijgehouden welke onderzoekers werken aan welke projecten.

\begin{lstlisting}[frame=single,language=java]
package be.pxl.demo.domain;

import jakarta.persistence.*;
import java.time.LocalDate;
import java.util.ArrayList;
import java.util.List;

@Entity
public class Project {
	
	// all other properties left out intentionally
	
	@ManyToMany(mappedBy = "projects")
	private List<Researcher> researchers = new ArrayList<>();


	public void addResearcher(Researcher researcher) {
		researchers.add(researcher);
	}


	// all other methods left out intentionally

}
\end{lstlisting}


\section{N + 1 query problem (optioneel)}

Stel dat we de entiteitsklasse Post en de entiteitsklasse PostComment hebben. Er is een unidirectionele veel-op-één relatie tussen PostComment en Post. Een PostComment behoort tot precies één Post, maar voor één Post kunnen er meerdere PostComments bestaan.

\begin{lstlisting}[frame=single,  language=java]
package be.pxl.demo.domain;

import jakarta.persistence.Entity;
import jakarta.persistence.GeneratedValue;
import jakarta.persistence.Id;

@Entity
public class Post {
	@Id
	@GeneratedValue
	private Long id;
	private String title;

	public Post() {
		// JPA only
	}

	public Post(String title) {
		this.title = title;
	}

	public Long getId() {
		return id;
	}

	public String getTitle() {
		return title;
	}
}
\end{lstlisting}

\begin{lstlisting}[frame=single,  language=java]
package be.pxl.demo.domain;

import jakarta.persistence.Entity;
import jakarta.persistence.GeneratedValue;
import jakarta.persistence.Id;
import jakarta.persistence.ManyToOne;

@Entity
public class PostComment {
	@Id
	@GeneratedValue
	private Long id;
	@ManyToOne
	private Post post;
	private String review;

	public PostComment() {
		// JPA only
	}

	public PostComment(Post post, String review) {
		this.post = post;
		this.review = review;
	}

	public Long getId() {
		return id;
	}

	public Post getPost() {
		return post;
	}

	public String getReview() {
		return review;
	}
}
\end{lstlisting}

We hebben een JpaRepository voor beide entiteitsklassen.
De PostService implementeert een methode om een post te maken, een methode voor het maken van reacties en een methode om alle reacties uit de database op te halen.

\begin{lstlisting}
package be.pxl.demo.service;

import be.pxl.demo.api.dto.PostCommentDTO;
import be.pxl.demo.domain.Post;
import be.pxl.demo.domain.PostComment;
import be.pxl.demo.exception.ResourceNotFoundException;
import be.pxl.demo.repository.PostCommentRepository;
import be.pxl.demo.repository.PostRepository;
import org.springframework.stereotype.Service;

import java.util.List;

@Service
public class PostService {

    private final PostCommentRepository postCommentRepository;
    private final PostRepository postRepository;

    public PostService(PostCommentRepository postCommentRepository,
                       PostRepository postRepository) {
        this.postCommentRepository = postCommentRepository;
        this.postRepository = postRepository;
    }

    public long createPost(String title) {
        return postRepository.save(new Post(title)).getId();
    }

    public void createPostComment(long postId, String review) {
        Post post = postRepository.findById(postId).orElseThrow(() -> new ResourceNotFoundException("Post", "id", postId));
        PostComment postComment = new PostComment(post, review);
        postCommentRepository.save(postComment);

    }

    public List<PostCommentDTO> findAll() {
        List<PostComment> allComments = postCommentRepository.findAll();
        return allComments.stream().map(pc -> new PostCommentDTO(pc.getPost().getTitle(), pc.getReview())).toList();
    }
}
\end{lstlisting}

Dan hebben we de PostController waar we twee endpoints implementeren. Eén endpoint voor het genereren van testdata en een ander endpoint voor het ophalen van alle reacties op de post.

\begin{lstlisting}
package be.pxl.demo.api;

import be.pxl.demo.api.dto.PostCommentDTO;
import be.pxl.demo.service.PostService;
import org.springframework.web.bind.annotation.GetMapping;
import org.springframework.web.bind.annotation.RequestMapping;
import org.springframework.web.bind.annotation.RestController;

import java.util.List;

@RestController
@RequestMapping("posts")
public class PostController {

    private final PostService postService;

    public PostController(PostService postService) {
        this.postService = postService;
    }

    @GetMapping("init")
    public void init() {
        long post1Id = postService.createPost("PXL'er Ward Lemmelijn kroont zich tot wereldkampioen indoorroeien.");
        long post2Id = postService.createPost("PXL naar finale in Cybersecurity challenge.");
        long post3Id = postService.createPost("Luister naar PXLRadio!!");

        postService.createPostComment(post1Id, "Schitterend ***");
        postService.createPostComment(post1Id, "Proficiat!");
        postService.createPostComment(post2Id, "Ik ben zeker van de partij!");
        postService.createPostComment(post2Id, "Ik hou van uitdagingen!");
        postService.createPostComment(post3Id, "Leuke muziek!");
        postService.createPostComment(post3Id, "Toffe radiozender!");
        postService.createPostComment(post3Id, "Zet die plaat af.");
    }

    @GetMapping
    public List<PostCommentDTO> getAllComments() {
        return postService.findAll();
    }
}
\end{lstlisting}

Om alle SQL instructies te kunnen analyseren enabelen we de property show-sql in het bestand application.properties.

\begin{lstlisting}
spring.jpa.show-sql=true
# Following property can be used to format the sql shown
# spring.jpa.properties.hibernate.format_sql=true 
\end{lstlisting}

Nadat we het endpoint hebben aangeroepen om testgegevens te genereren,  halen we alle reacties op uit de database.
In de logbestanden vind je de volgende sql instructies.

\begin{lstlisting}
Hibernate: select p1_0.id,p1_0.post_id,p1_0.review from post_comment p1_0
Hibernate: select p1_0.id,p1_0.title from post p1_0 where p1_0.id=?
Hibernate: select p1_0.id,p1_0.title from post p1_0 where p1_0.id=?
Hibernate: select p1_0.id,p1_0.title from post p1_0 where p1_0.id=?
\end{lstlisting}

Nadat de reacties uit de database zijn opgehaald, wordt de originele post bij de reactie ook opgehaald. Dus, als je N originele posts hebt,  worden er N+1-query's uitgevoerd. Dat is niet efficiënt!

Je zou dit probleem moeten aanpakken door de query voor het ophalen van alle postreacties te herschrijven.

\begin{lstlisting}
package be.pxl.demo.repository;

import be.pxl.demo.domain.PostComment;
import org.springframework.data.jpa.repository.EntityGraph;
import org.springframework.data.jpa.repository.JpaRepository;

import java.util.List;

public interface PostCommentRepository extends JpaRepository<PostComment, Long> {

    @Override
    @EntityGraph(
            type = EntityGraph.EntityGraphType.FETCH,
            attributePaths = {
                    "post"
            }
    )
    List<PostComment> findAll();
}
\end{lstlisting}

Bij het ophalen van reacties met de methode findAll(), zorgt de @EntityGraph annotatie ervoor dat Spring Data JPA en Hibernate de geassocieerde entiteiten ophalen die zijn opgenomen in attributePaths.

Er wordt nu slechts één query uitgevoerd.

\begin{lstlisting}
Hibernate: select p1_0.id,p2_0.id,p2_0.title,p1_0.review from post_comment p1_0 left join post p2_0 on p2_0.id=p1_0.post_id
\end{lstlisting}

Het is tijdens de development-fase steeds belangrijk om de gegenereerde query's te controleren.  Het is een goede gewoonte om SQL-output in te schakelen tijdens het werken aan een Spring Boot-project, gewoon als een sanity check.
Je zult af en toe problemen vinden waarvan je je niet bewust was, door het controleren van de SQL-output.

Een uitgebreid voorbeeld van het N+1 query-probleem en de oplossing is te vinden in deze post: \url{https://tech.asimio.net/2020/11/06/Preventing-N-plus-1-select-problem-using-Spring-Data-JPA-EntityGraph.html}.

\section{Lazy loading en OSIV}

De entiteiten in een @OneToMany en @ManyToMany relatie worden op een 'lazy' manier geladen. Het laden van de data gebeurt enkel wanneer het nodig is.  \textbf{Lazy loading} verbetert de performantie van de applicatie.  Je hebt echter een transactie (of sessie) nodig om de geassocieerde data uit de database op te halen.
Spring Boot schakelt echter standaard een mechanisme in dat OSIV (Open Session in View) wordt genoemd. In Spring Boot applicaties die JPA gebruiken, worden transacties gecreëerd zelfs voordat requests bij de restcontroller aankomen, zelfs als het verzoek helemaal geen databasebewerkingen uitvoert. De Spring OpenSessionInViewInterceptor klasse beheert deze sessies.  OSIV is een slecht idee vanuit een prestatie- en schaalbaarheidsperspectief.

Zorg er dus voor dat je in het configuratiebestand application.properties de volgende eigenschap toevoegt:


\begin{lstlisting}
spring.jpa.open-in-view=false
\end{lstlisting}

Wat gebeurt er als je gegevens ophaalt van een 'lazy loaded' associatie zonder een actieve transactie of sessie?

Laten we starten met de associatie tussen Post en PostComment bidirectioneel te maken.

\begin{lstlisting}
package be.pxl.demo.domain;

import jakarta.persistence.Entity;
import jakarta.persistence.GeneratedValue;
import jakarta.persistence.Id;
import jakarta.persistence.OneToMany;

import java.util.List;

@Entity
public class Post {
	@Id
	@GeneratedValue
	private Long id;
	private String title;
	@OneToMany(mappedBy = "post")
	private List<PostComment> commentList;

	public Post() {
		// JPA only
	}

	public Post(String title) {
		this.title = title;
	}

	public Long getId() {
		return id;
	}

	public String getTitle() {
		return title;
	}

	public List<PostComment> getCommentList() {
		return commentList;
	}
}
\end{lstlisting}

\begin{lstlisting}
package be.pxl.demo.service;

import be.pxl.demo.api.dto.PostCommentDTO;
import be.pxl.demo.api.dto.PostDTO;
import be.pxl.demo.domain.Post;
import be.pxl.demo.domain.PostComment;
import be.pxl.demo.exception.ResourceNotFoundException;
import be.pxl.demo.repository.PostCommentRepository;
import be.pxl.demo.repository.PostRepository;
import org.springframework.stereotype.Service;

import java.util.List;

@Service
public class PostService {

    private final PostCommentRepository postCommentRepository;
    private final PostRepository postRepository;

    public PostService(PostCommentRepository postCommentRepository,
                       PostRepository postRepository) {
        this.postCommentRepository = postCommentRepository;
        this.postRepository = postRepository;
    }

   

    public PostDTO getPost(long postId) {
        Post post = postRepository.findById(postId)
               .orElseThrow(() -> new ResourceNotFoundException("Post", "id", postId));
        return new PostDTO(post.getTitle(), 
                 post.getCommentList().stream().map(PostComment::getReview).toList());
    }
}
\end{lstlisting}

\begin{lstlisting}
package be.pxl.demo.api.dto;

import java.util.List;

public record PostDTO(String title, List<String> review) {
}
\end{lstlisting}


\begin{lstlisting}
package be.pxl.demo.api;

import be.pxl.demo.api.dto.PostCommentDTO;
import be.pxl.demo.api.dto.PostDTO;
import be.pxl.demo.service.PostService;
import org.springframework.web.bind.annotation.GetMapping;
import org.springframework.web.bind.annotation.PathVariable;
import org.springframework.web.bind.annotation.RequestMapping;
import org.springframework.web.bind.annotation.RestController;

import java.util.List;

@RestController
@RequestMapping("posts")
public class PostController {

    private final PostService postService;

    public PostController(PostService postService) {
        this.postService = postService;
    }

    @GetMapping(path = "{postId}")
    public PostDTO getPost(@PathVariable long postId) {
        return postService.getPost(postId);
    }
}
\end{lstlisting}

Als we het endpoint aanroepen om een post op te halen krijgen we de volgende LazyIntializationException: 

\begin{lstlisting}[basicstyle=\tiny]
org.hibernate.LazyInitializationException: failed to lazily initialize a collection of role: be.pxl.demo.domain.Post.commentList: could not initialize proxy - no Session
	at org.hibernate.collection.spi.AbstractPersistentCollection.throwLazyInitializationException(AbstractPersistentCollection.java:635) ~[hibernate-core-6.1.7.Final.jar:6.1.7.Final]
	at org.hibernate.collection.spi.AbstractPersistentCollection.withTemporarySessionIfNeeded(AbstractPersistentCollection.java:218) ~[hibernate-core-6.1.7.Final.jar:6.1.7.Final]
	at org.hibernate.collection.spi.AbstractPersistentCollection.initialize(AbstractPersistentCollection.java:615) ~[hibernate-core-6.1.7.Final.jar:6.1.7.Final]
	at org.hibernate.collection.spi.AbstractPersistentCollection.read(AbstractPersistentCollection.java:136) ~[hibernate-core-6.1.7.Final.jar:6.1.7.Final]
	at org.hibernate.collection.spi.PersistentBag.iterator(PersistentBag.java:366) ~[hibernate-core-6.1.7.Final.jar:6.1.7.Final]
	at java.base/java.util.Spliterators$IteratorSpliterator.estimateSize(Spliterators.java:1865) ~[na:na]
	at java.base/java.util.Spliterator.getExactSizeIfKnown(Spliterator.java:414) ~[na:na]
	at java.base/java.util.stream.AbstractPipeline.exactOutputSizeIfKnown(AbstractPipeline.java:470) ~[na:na]
	at java.base/java.util.stream.AbstractPipeline.evaluate(AbstractPipeline.java:574) ~[na:na]
	at java.base/java.util.stream.AbstractPipeline.evaluateToArrayNode(AbstractPipeline.java:260) ~[na:na]
	at java.base/java.util.stream.ReferencePipeline.toArray(ReferencePipeline.java:616) ~[na:na]
	at java.base/java.util.stream.ReferencePipeline.toArray(ReferencePipeline.java:622) ~[na:na]
	at java.base/java.util.stream.ReferencePipeline.toList(ReferencePipeline.java:627) ~[na:na]
	at be.pxl.demo.service.PostService.getPost(PostService.java:39) ~[classes/:na]
	...
\end{lstlisting}

In de methode getPost() in onze PostService worden de reacties op onze post 'lazy loaded', maar op dat moment is er geen sessie (of transactie) actief.
De juiste oplossing is om een transactie te definiëren voor de methode getPost() in de PostService. Het is ook mogelijk om de relatie als EAGER in plaats van LAZY te definiëren, maar dit is zelden een goede oplossing.


\begin{lstlisting}
package be.pxl.demo.service;

import be.pxl.demo.api.dto.PostCommentDTO;
import be.pxl.demo.api.dto.PostDTO;
import be.pxl.demo.domain.Post;
import be.pxl.demo.domain.PostComment;
import be.pxl.demo.exception.ResourceNotFoundException;
import be.pxl.demo.repository.PostCommentRepository;
import be.pxl.demo.repository.PostRepository;
import org.springframework.stereotype.Service;

import java.util.List;

@Service
public class PostService {

    private final PostCommentRepository postCommentRepository;
    private final PostRepository postRepository;

    public PostService(PostCommentRepository postCommentRepository,
                       PostRepository postRepository) {
        this.postCommentRepository = postCommentRepository;
        this.postRepository = postRepository;
    }

   
	@Transactional(readOnly = true)
    public PostDTO getPost(long postId) {
        Post post = postRepository.findById(postId)
               .orElseThrow(() -> new ResourceNotFoundException("Post", "id", postId));
        return new PostDTO(post.getTitle(), 
                 post.getCommentList().stream().map(PostComment::getReview).toList());
    }
}
\end{lstlisting}

Meer info over dit onderwerp vind je in de volgende artikels:
\begin{itemize} 
\item \url{
https://www.baeldung.com/spring-open-session-in-view}
\item \url{https://techiemanas.wordpress.com/2020/12/01/spring-open-session-in-view-osiv/}

\end{itemize}

