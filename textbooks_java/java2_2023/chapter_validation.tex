\chapter{Spring Validation}
    
\fcolorbox{black}[HTML]{E9F0E9}{\parbox{\textwidth}{%
\noindent \textbf{Learning goals}\\
De junior-collega
\begin{enumerate}[nolistsep]
\item kan gebruikmaken van Spring Validation in een project
\end{enumerate}}}


When Spring receives an HTTP request, the data must be validated. 
Spring validation offers  standard predefined validators.  

\begin{lstlisting}
<dependency>
	<groupId>org.springframework.boot</groupId>
	<artifactId>spring-boot-starter-validation</artifactId>
</dependency>
\end{lstlisting}

When Spring finds an argument of an HTTP method annotated with @Valid, it automatically validates the argument.  In this example, the SuperheroRequest is validated. 
The @NotBlank annotation checks that a String's trimmed length is not empty.
The @Size annotation validates that the length of the string is between min and max.

Spring also offers the annotation @Validated. This is used for group validation which is not in the scope of this course. \url{https://blog.codeleak.pl/2014/08/validation-groups-in-spring-mvc.html}

Here is a list of the most common validation annotations.

\begin{tabularx}{\textwidth}{|l|X|}
\hline
@NotNull & to say that a field must not be null. \\
@NotEmpty & to say that a list field must not empty. \\
@NotBlank & to say that a string field must not be the empty string. \\
@Min and @Max & to say that a numerical field is only valid when it’s value is above or below a certain value.\\
@Size & to validate the length of a string field. \\
@Email & to say that a string field must be a valid email address.\\
\hline
\end{tabularx}

\begin{lstlisting}[language=java, frame=single]
package be.pxl.demo.domain;

import com.fasterxml.jackson.annotation.JsonProperty;
import jakarta.validation.constraints.Min;
import jakarta.validation.constraints.NotBlank;
import jakarta.validation.constraints.NotNull;

public class Song {
	@NotBlank
	private String title;
	@NotBlank
	private String artist;
	@JsonProperty("duration_seconds")
	@Min(value = 1, message = "To must be greater than zero")
	private int durationSeconds;
	@NotNull
	private Genre genre;
	
	...
}
\end{lstlisting}

It is also possible to use the validation constraints with the fields of java classes.

When the target argument fails the validation, Spring Boot throws a MethodArgumentNotValidException exception.