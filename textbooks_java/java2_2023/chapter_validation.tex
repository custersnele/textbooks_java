\chapter{Spring Validation}
    
\fcolorbox{black}[HTML]{E9F0E9}{\parbox{\textwidth}{%
\noindent \textbf{Learning goals}\\
De junior-collega
\begin{enumerate}[nolistsep]
\item kan gebruikmaken van Spring Validation
\item kan de correcte validatieregels gebruiken
\end{enumerate}}}

Wanneer Spring Boot een HTTP-verzoek ontvangt, is het belangrijk om de gegevens die met dat verzoek zijn meegestuurd te valideren.  Spring Boot heeft voorgedefinieerde validatoren waardoor het controleren van veelvoorkomende regels eenvoudig kan gebeuren.  Zo kan Spring Boot bijvoorbeeld heel eenvoudig controleren of een e-mailadres het juiste formaat heeft of een getal binnen een bepaald bereik valt,.
In Spring Boot is het uiteraard ook mogelijk om aangepaste of eigen validatieregels te programmeren.

Om Spring Validation te gebruiken voeg je een dependency toe aan het project.

\begin{lstlisting}
<dependency>
	<groupId>org.springframework.boot</groupId>
	<artifactId>spring-boot-starter-validation</artifactId>
</dependency>
\end{lstlisting}

Hier is een lijst van de meest gebruikte annotaties.

\begin{tabularx}{\textwidth}{|l|X|}
\hline
@NotNull & om aan te duiden dat een referentie niet null mag zijn. \\
@NotEmpty & om aan te duiden dat een list elementen moet bevatten. \\
@NotBlank & om aan te duiden dat een string niet leeg mag zijn. \\
@Min and @Max & om aan te duiden dat een numerieke waarde enkel geldig is als de waarde groter of kleiner is dan een gegeven waarde.\\
@Size &  om de lengte van een string te valideren. \\
@Email & om te controleren of een string een geldig e-mailadres bevat.\\
\hline
\end{tabularx}

\begin{lstlisting}[language=java, frame=single]
package be.pxl.demo.domain;

import com.fasterxml.jackson.annotation.JsonProperty;
import jakarta.validation.constraints.Min;
import jakarta.validation.constraints.NotBlank;
import jakarta.validation.constraints.NotNull;

public class Song {
	@NotBlank
	private String title;
	@NotBlank
	private String artist;
	@JsonProperty("duration_seconds")
	@Min(value = 1, message = "Must be greater than zero")
	private int durationSeconds;
	@NotNull
	private Genre genre;
	
	...
}
\end{lstlisting}

Wanneer Spring Boot een argument vindt dat is geannoteerd met @Valid,  wordt het automatisch gevalideerd.

\begin{lstlisting}
@RestController
@RequestMapping("/playlist/songs")
public class MusicPlaylistController {

	private static final Logger LOGGER = LoggerFactory.getLogger(MusicPlaylistController.class);
	private final MusicPlaylistService musicPlaylistService;

	@Autowired
	public MusicPlaylistController(MusicPlaylistService musicPlaylistService) {
		this.musicPlaylistService = musicPlaylistService;
	}

	@PostMapping
	public void addSong(@RequestBody @Valid Song song) {
		if (LOGGER.isInfoEnabled()) {
			LOGGER.info("Adding song [" + song.getTitle() + "]");
		}
		musicPlaylistService.addSong(song);
	}
   ...
}
\end{lstlisting}

Wanneer het argument niet voldoet aan de regels dan gooit Spring Boot een exception van de klasse MethodArgumentNotValidException.

Sinds Spring Boot versie 2.3 moet je de eigenschap server.error.include-message de waarde always geven om de foutboodschappen te kunnen zien in het response. Spring Boot verbergt namelijk het foutmelding in de respons om het lekken van gevoelige informatie te voorkomen. 

De foutmelding blijft ondanks in de instelling eerder cryptisch.  Stel dat ik de REST API gebruik om een lied toe te voegen met de volgende gegevens:

\begin{verbatim}
{
  "title": "",
  "artist": "Bruno Mars",
  "duration_seconds": -2,
  "genre": "OTHER"
}
\end{verbatim}

Dan krijg je het volgende respons.

\begin{verbatim}
{
	"timestamp": "2023-09-23T13:48:50.490+00:00",
	"status": 400,
	"error": "Bad Request",
	"message": "Validation failed for object='song'. Error count: 2",
	"path": "/playlist/songs"
}
\end{verbatim}

Meer gegevens krijg ik te zien als ik de eigenschap server.error.include-binding-errors=always toevoeg in application.properties. Voor bovenstaande RequestBody krijg je dan de volgende respons. Er wordt op dat ogenblik al veel interne informatie gelekt.

\begin{lstlisting}[language=json]
{
	"timestamp": "2023-09-23T13:52:38.028+00:00",
	"status": 400,
	"error": "Bad Request",
	"message": "Validation failed for object='song'. Error count: 2",
	"errors": [
		{
			"codes": [
				"Min.song.durationSeconds",
				"Min.durationSeconds",
				"Min.int",
				"Min"
			],
			"arguments": [
				{
					"codes": [
						"song.durationSeconds",
						"durationSeconds"
					],
					"arguments": null,
					"defaultMessage": "durationSeconds",
					"code": "durationSeconds"
				},
				10
			],
			"defaultMessage": "Invalid duration_seconds: must equal or exceed 10.",
			"objectName": "song",
			"field": "durationSeconds",
			"rejectedValue": -2,
			"bindingFailure": false,
			"code": "Min"
		},
		{
			"codes": [
				"NotBlank.song.title",
				"NotBlank.title",
				"NotBlank.java.lang.String",
				"NotBlank"
			],
			"arguments": [
				{
					"codes": [
						"song.title",
						"title"
					],
					"arguments": null,
					"defaultMessage": "title",
					"code": "title"
				}
			],
			"defaultMessage": "must not be blank",
			"objectName": "song",
			"field": "title",
			"rejectedValue": "",
			"bindingFailure": false,
			"code": "NotBlank"
		}
	],
	"path": "/playlist/songs"
}
\end{lstlisting}

De best practice om de MethodArgumentNotValidException af te handelen is door gebruik te maken van een exception handler.  Hierbij gebruiken we de annotaties @RestControllerAdvice en @ExceptionHandler. 

\begin{lstlisting}
package be.pxl.demo.config;

import org.springframework.http.HttpStatus;
import org.springframework.http.ResponseEntity;
import org.springframework.validation.ObjectError;
import org.springframework.web.bind.MethodArgumentNotValidException;
import org.springframework.web.bind.annotation.ExceptionHandler;
import org.springframework.web.bind.annotation.RestControllerAdvice;

import java.util.ArrayList;
import java.util.List;

@RestControllerAdvice
public class GlobalExceptionHandler {

	@ExceptionHandler(MethodArgumentNotValidException.class)
	public ResponseEntity<List<String>> handleValidationErrors(MethodArgumentNotValidException ex) {
		List<String> errors = new ArrayList<>();
		for (ObjectError error: ex.getBindingResult().getAllErrors()) {
			errors.add(error.getDefaultMessage());
		}
		return new ResponseEntity<>(errors, HttpStatus.BAD_REQUEST);
	}
}
\end{lstlisting}


De annotatie @RestControllerAdvice geeft aan dat deze klasse verantwoordelijk is voor het afhandelen van exceptions. Alle exception worden onderschept en indien er voor het type exception een handler is gedefinieerd, zal de bijhorende code worden uitgevoerd. 

In deze klasse is er enkel een methode voorzien voor het afhandelen van MethodArgumentNotValidExceptions. De methode herkennen we aan de annotatie @ExceptionHandler. Binnen de methode wordt een lijst met de foutmeldingen (errors) gegenereerd.

We passen de foutboodschap bij iedere validatie ook best aan.  Gebruik een duidelijke foutboodschap om de gebruiker duidelijk te maken welke gegevens hij moet doorgeven.
Als je een meertalige applicatie hebt, kan je best een key (zoals title.not.blank) doorgeven aan de frontend zodat de eigenlijk foutboodschap in de juiste taal getoond kan worden.


\begin{lstlisting}
	@NotBlank(message = "title.not.blank")
	private String title;
	@Min(value = 10, message = "Invalid duration_seconds: must be equal to or exceed 10.")
	private int durationSeconds;
\end{lstlisting}

Er wordt in het geval van een MethodArgumentNotValidException een HTTP status 400 met de volgende response body gegeven:
\begin{verbatim}
[
	"Invalid duration_seconds: must be equal to or exceed 10.",
	"title.not.blank",
	"Please provide artist name."
]
\end{verbatim}


\begin{oefening}
\textbf{Huizenjacht}

We voeren een aantal aanpassingen uit aan de klasse House.
\begin{itemize}
\item Voeg de volgende eigenschappen toe:
\begin{itemize}
\item area: oppervlakte van de woning in m\textsuperscript{2}
\item epcScore: waarde uit de opsomming EPCScore.
\end{itemize}
\item Verwijder de eigenschap price. De prijs is nu een berekende waarde. Vermenigvuldig de oppervlakte met de basisprijs per m\textsuperscript{2} en het percentage behorende bij de epc score.
De huidige basisprijs per m\textsuperscript{2} is 2356.75.  Voor huizen in Hasselt en Genk is er een meerprijs van 25\%. 
\item Voeg de methode markAsSold() toe.  Deze methode verandert de status van het huis van FOR\_SALE naar SOLD. Wanneer het huis al verkocht is gooi je een IllegalStateException op.
\item Pas vervolgens de POST en PUT requests aan. Bij het toevoegen van een huis zijn de volgende velden verplicht: code, name, area (minstens 50m\textsuperscript{2}), epc-score en city. 
Bij het updaten kan je enkel volgende velden aanpassen: name, area (minstens 50m\textsuperscript{2}), city en epc-score.
\item Voorzie een aangepast endpoint /houses/\{code\}/sold om een huis als verkocht te registreren.
\end{itemize}

Test alle endpoints grondig uit.

\begin{lstlisting}[language=java]
public enum EPCScore {
	A_PLUS(1.5),
	A(1.2),
	B(1),
	C(0.98),
	D(0.90),
	E(0.80),
	F(0.75);

	private double percentage;

	EPCScore(double percentage) {
		this.percentage = percentage;
	}

	public double getPercentage() {
		return percentage;
	}
}
\end{lstlisting}

\end{oefening}

